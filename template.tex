% !TeX spellcheck = en_US
\documentclass[11pt]{article}
\usepackage[a4paper,bindingoffset=0.2in,%
			left=1in,right=1in,top=1in,bottom=1in,%
			footskip=.25in]{geometry}
\bibliographystyle{ieeetr}

% --------------------------------------------------------------------
%                        User Defined Packages
% --------------------------------------------------------------------
% For truncating long sentences
% Turn "This text has been truncated"
%   to "This text has been trun-..."
\usepackage[hyphenate]{truncate}
% For including figures
\usepackage{graphicx}
% For arranging subfigures flexibly
\usepackage{subcaption,array}
% For setting the font and size of the figure captions
\usepackage{caption}
\captionsetup{width=0.8\textwidth}
\captionsetup[table]{labelsep=newline,name=TABLE,textfont=sc}
\captionsetup[figure]{labelsep=period,name=Fig.,font=scriptsize}
%     - Tables use Roman numbering
\renewcommand\thetable{\Roman{table}}
% For adding footnote to a table
\usepackage[flushleft]{threeparttable}
% For the diagonal slash in a table
\usepackage{diagbox}
% For \multirow
\usepackage{multirow}
% \toprule, \midrule, and \bottomrule
\usepackage{booktabs}
% New column types that accept width. e.g. | L{2cm} | C{1cm} |
\newcolumntype{L}[1]{>{\raggedright\let\newline\\\arraybackslash\hspace{0pt}}m{#1}}
\newcolumntype{C}[1]{>{\centering\let\newline\\\arraybackslash\hspace{0pt}}m{#1}}
\newcolumntype{R}[1]{>{\raggedleft\let\newline\\\arraybackslash\hspace{0pt}}m{#1}}
% For the \scaleto text scaling
\usepackage{scalerel}
% Formating the appendix sections
\usepackage[title]{appendix}
% Link icon \faExternalLink
\usepackage{fontawesome5}
% For \notebox, \tipbox, \warningbox, \cautionbox, \importantbox, and
% For noteblock, tipblock, warningblock, cautionblock, importantblock
\usepackage{awesomebox}
% Fit table to the page width automatically
\usepackage{adjustbox}
% Provides easy color coding. e.g. {blue!80!black} means blue*0.8 + black*0.2
%\usepackage[table]{xcolor}  % Loads also »colortbl«
\usepackage{xcolor}
% For rendering algorithm pseudo code
\usepackage{amsmath}
\usepackage{algorithm}
\usepackage[noend]{algpseudocode}
% For defining "formal" environment as a fancy quotation block
\usepackage{framed}
% For the adjustwidth environment, used in "formal" environment
\usepackage[strict]{changepage}
% For concecutive in-text citations to follow the "[1], [3]-[5]" style
%\usepackage[noadjust]{cite} % The old hacky way to make "cite" package work
%\renewcommand\citeform[1]{[#1]}
%\renewcommand\citeleft{}
%\renewcommand\citeright{}
\usepackage[numbers,sort&compress]{natbib} % \cite[p.~10]{refName} to add additional information 

% ------------------------------
%    Packages Kept at Bottom
% ------------------------------
% [Bottom] Some suggest to load hyperref last
% 	colorlinks: Colours links instead of ugly boxes
% 	linkcolor: Colour of internal links. [red/blue/green]
% 	urlcolor: Colour for external links
% 	citecolor: Colour of citations
\usepackage[colorlinks,
pagebackref,
linkcolor={blue!80!black},
anchorcolor={blue!80!black},
citecolor={blue!80!black},
urlcolor={blue!80!black}
]{hyperref}
% Customize the backref command of hyperref. Backref link from the references.
\renewcommand\backrefxxx[3]{%
	\hyperlink{page.#1}{$\uparrow$#1}%
}
\newcommand\myBlogUrl{http://schenax.student.ust.hk}
% A macro to show the date of last update
\newcommand{\updateinfo}[1][\today]{\par\vfill\hfill{\scriptsize\color{gray}Last updated on #1}}
% Paper ratings icons
\newcommand{\bad}{\textcolor[HTML]{EC6B56}{\faIcon[regular]{frown}}}  % A paper not worth reading again ---- red
\newcommand{\soso}{\textcolor[HTML]{FFC154}{\faIcon[regular]{grin-alt}}}  % An interesting paper ---- yellow
\newcommand{\good}{\textcolor[HTML]{47B39C}{\faIcon[regular]{laugh-beam}}}  % A paper worth reading many times ---- green
% [Bottom] For spliting a large Latex project into sub-projects (e.g. chapters)
\usepackage{subfiles}

% --------------------------------------------------------------------
%                            User Environment
% --------------------------------------------------------------------
% "formal" environment -- a fancy quotation block
\definecolor{formalgray}{gray}{0.8}
\definecolor{formalshadegray}{gray}{0.9}
\definecolor{formalshade}{rgb}{0.95,0.95,1}
\newenvironment{formal}{%
	\def\FrameCommand{%
		\hspace{1pt}%
		{\color{formalgray}\vrule width 2pt}%
		{\color{formalshade}\vrule width 4pt}%
		\colorbox{formalshade}%
	}%
	\MakeFramed{\advance\hsize-\width\FrameRestore}%
	\noindent\hspace{-4.55pt}% disable indenting first paragraph
	\begin{adjustwidth}{}{7pt}%
		\vspace{2pt}\vspace{2pt}%
	}
	{%
		\vspace{2pt}
	\end{adjustwidth}\endMakeFramed%
}

% -----------------------------
%     Experimental Packages
% -----------------------------
% \usepackage{tabularx}
% Used just in this template to generate dummy content
\usepackage{blindtext}


\title{Research Note Template}
\author{Shixi Chen}
\date{December 18, 2020}

\begin{document}

% Start roman numbering
\pagenumbering{roman}
\maketitle

%\begin{abstract}
%The \texttt{outlines} package defines the \texttt{outline} environment,
%that allows outline-style indented lists with freely mixed levels up
%to four levels deep.  It replaces the nested "begin"/"end" pairs by
%different item tags "\1" to "\4" for each nesting level.  This is very
%convenient in cases where nested lists are used a lot, such as for to-do
%lists or presentation slides.
%\end{abstract}

% e.g. 2 -- TOC displays part,chapters,sections,subsections
%  -1 = \part
% 	0 = \chapter
% 	1 = \section
% 	2 = \subsection
% 	3 = \subsubsection
% 	4 = \paragraph
% 	5 = \subparagraph
\addtocontents{toc}{\protect\setcounter{tocdepth}{3}}
\tableofcontents
% Render the last compile time. With arg: \updateinfo[December 18, 2020]
\updateinfo

\newpage
% Switch to normal numbering
\pagenumbering{arabic}
\section{Your Section One}

\blindtext

\begin{formal}
\textbf{Q}: How can I place a continuous discussion thread that tracks my thinking process?

\noindent\textbf{A}: Write your thinking in a Q\&A format inside a \textbackslash begin\{formal\} \textbackslash\{formal\} block. 
\end{formal}

\subsection{Your Subsection}

\blindtext

\begin{figure}[!hbt]
	\centering
	\begin{subfigure}[b]{0.38\textwidth}
		\includegraphics[width=\textwidth]{example-image-a}
		\caption{}
		\label{fig:exampleImageA}		% The \label should always be after \caption
		\includegraphics[width=\textwidth]{example-image-b}
		\caption{}
		\label{fig:exampleImageB}		% The \label should always be after \caption
	\end{subfigure}
	\begin{subfigure}[b]{0.6\textwidth}
		\centering
		\includegraphics[width=\textwidth,angle=90]{example-image-c}
		\caption{}
		\label{fig:exampleImageC}		% The \label should always be after \caption
	\end{subfigure}
	\caption{Explanation of the figure presentation. (a) Evidence A. (b) Evidence B. (c) Evidence C.}
	\label{fig:exampleImages}		% The \label should always be after \caption
\end{figure}

\blindtext

\begin{noteblock}
Some information goes here

And some other information goes here
\end{noteblock}

\blindtext

\begin{table}[!htb]
	\centering
	%	\renewcommand{\arraystretch}{1.5}
	\caption{Electrical Power Model}
	\label{tab:electricalPwrModel}		% The \label should always be after \caption
	\begin{adjustbox}{max width=\textwidth}  % Make the table fit text width
		\begin{threeparttable}
			\begin{tabular}{l r r r}
				\toprule
				\textbf{Parameter} 					  & \textbf{Dynamic Power}    & \textbf{Static Power} 			& \textbf{Note}					 \\ \midrule
				core@4GHz + L1 I/D Cache 			  & 1.50185 W  			      & 0.131346 W 		   				& \multirow[c]{7}{*}{McPAT 7nm}  \\
				% [vpos]{nrows}{width}
				\multirow[t]{2}{*}{512KB L2 Slice}    & Read: 0.0226595 nJ/op     & \multirow[t]{2}{*}{0.0238705 W} &            					 \\
				                                      & Write: 0.0226595 nJ/op    &                                 &                                \\
				\multirow[t]{2}{*}{4MB L3 Slice}      & Read: 0.080728 nJ/op      & \multirow[t]{2}{*}{0.189787 W}  &                                \\
				                                      & Write: 0.080728 nJ/op     &            						&								 \\
				router                                & 0.09 pJ/bit               & 0.11 W                          &                                \\
				buffer                                & 0.064 pJ/bit              & 0.0002 W                        &                                \\ \midrule
				intra-die elink (16-Byte wide, uni-)  & 0.0768 pJ/bit             & 0.017 W                         &                                \\
				inter-die elink                       & 1.02 pJ/bit               & \textcolor{red}{0 W, no report} & \cite{Something}               \\
				inter-chip elink \tnote{a}            & 4.71 pJ/bit               & \textcolor{red}{0 W, no report} & OEIL                           \\ \midrule
				memory controller                     & 0.0391783 W               & 0.00194085 W                    & McPAT                          \\ \midrule
				memory                                & 0.24 W                    & 0.075 W                         & MICRON DDR4 8GB                \\
				                                      &                           &                                 & @(9\% read, 3\%write)          \\ \bottomrule
			\end{tabular}
			\begin{tablenotes}
				\newcommand{\CORNERSTONE}{\href{run:./appendix/CORNERSTONE-Standard-Components-Library-SOI.pdf}{CORNERSTONE}}
				\scriptsize
				\item [a] Highly dependent on length, working frequency, etc. Value derived from OEIL's default device parameters, check the manual for details.
			\end{tablenotes}
		\end{threeparttable}
	\end{adjustbox}
\end{table}

\subsubsection{Your Subsubsection}

\paragraph{Object1} is the focus of this paragraph. \blindtext

\paragraph{Object2} is the focus of this paragraph. \blindtext

\subsection{Literature}
% From this point on, hide sub-sub-sections in the ToC
\addtocontents{toc}{\protect\setcounter{tocdepth}{2}}

\cite{Boney96, MG, HK, Pan, Something} has reported promising results on phenomena A, B, C while \cite{Something} provides more evidence to support the hypothesis. Particularly, \citep[Sec.~III]{Something} explains their relationship in great details and proposes a new model called WWW.

\begin{table}[!htb]
\centering
\label{tab:literature}		% The \label should always be after \caption
% \resizebox{\textwidth}{!}{  % Make the table fit text width
\begin{adjustbox}{max width=\textwidth}  % Make the table fit text width
\begin{threeparttable}
\newcommand{\titlelength}{230pt}  % Defines the display length of a title
\begin{tabular}{ l l l l l }
	\toprule
	\textbf{Title}                                                                                          & \textbf{Type} & \textbf{Date} & \textbf{Group} & \textbf{Page}                                  \\ \midrule
	\truncate{\titlelength}{\good A very very very very very very very very very very long title}\cite{Boney96}   & Note1         & 2008          & Boney96        & \ref{lit:labelOfYourSubsubsection1} \\
	\truncate{\titlelength}{\good A very very very very very very very very very very long title}\cite{MG}        & Note2         & 2009          & MG             & \ref{lit:labelOfYourSubsubsection2} \\
	\truncate{\titlelength}{\soso A very very very very very very very very very very long title}\cite{HK}        & Note3         & 2010          & HK             & \ref{lit:labelOfYourSubsubsection3} \\
	\truncate{\titlelength}{\good A very very very very very very very very very very long title}\cite{Pan}       & Note4         & 2011          & Pan            & \ref{lit:labelOfYourSubsubsection4} \\
	\truncate{\titlelength}{\bad A very very very very very very very very very very long title}\cite{Something}  & Note5         & 2012          & Shixi          & \ref{lit:labelOfYourSubsubsection5} \\ \bottomrule
\end{tabular}
\end{threeparttable}
\end{adjustbox}
\end{table}

\subsubsection{A very very very very very very very very very very long title \texorpdfstring{\hyperref[tab:literature]{\footnotesize \hfill\faIcon[regular]{arrow-alt-circle-up}}}{}}
% \texorpdfstring is used to generate proper PDF indices
\label{lit:labelOfYourSubsubsection1}		% The \label should always be after \caption
{
	\tiny
	Authors: Dana Vantrease, Robert Schreiber, Matteo Monchiero, Moray McLaren, Norman P. Jouppi, Marco Fiorentino, Al Davis, Nathan Binkert, Raymond G. Beausoleil, Jung Ho Ahn \cite{Boney96}
	\par  % \par is the paragraph end and makes LaTeX (re-)compute the line spacing with \tiny
}

\blindtext

\subsubsection{A very very very very very very very very very very long title \texorpdfstring{\hyperref[tab:literature]{\footnotesize \hfill\faIcon[regular]{arrow-alt-circle-up}}}{}}
% \texorpdfstring is used to generate proper PDF indices
\label{lit:labelOfYourSubsubsection2}		% The \label should always be after \caption
{
	\tiny
	Authors: Dana Vantrease, Robert Schreiber, Matteo Monchiero, Moray McLaren, Norman P. Jouppi, Marco Fiorentino, Al Davis, Nathan Binkert, Raymond G. Beausoleil, Jung Ho Ahn \cite{MG}
	\par  % \par is the paragraph end and makes LaTeX (re-)compute the line spacing with \tiny
}

\blindtext

\subfile{latex/pseudocode.tex}

\subsubsection{A very very very very very very very very very very long title \texorpdfstring{\hyperref[tab:literature]{\footnotesize \hfill\faIcon[regular]{arrow-alt-circle-up}}}{}}
% \texorpdfstring is used to generate proper PDF indices
\label{lit:labelOfYourSubsubsection3}		% The \label should always be after \caption
{
	\tiny
	Authors: Dana Vantrease, Robert Schreiber, Matteo Monchiero, Moray McLaren, Norman P. Jouppi, Marco Fiorentino, Al Davis, Nathan Binkert, Raymond G. Beausoleil, Jung Ho Ahn \cite{HK}
	\par  % \par is the paragraph end and makes LaTeX (re-)compute the line spacing with \tiny
}

\blindtext

\subsubsection{A very very very very very very very very very very long title \texorpdfstring{\hyperref[tab:literature]{\footnotesize \hfill\faIcon[regular]{arrow-alt-circle-up}}}{}}
% \texorpdfstring is used to generate proper PDF indices
\label{lit:labelOfYourSubsubsection4}		% The \label should always be after \caption
{
	\tiny
	Authors: Dana Vantrease, Robert Schreiber, Matteo Monchiero, Moray McLaren, Norman P. Jouppi, Marco Fiorentino, Al Davis, Nathan Binkert, Raymond G. Beausoleil, Jung Ho Ahn \cite{Pan}
	\par  % \par is the paragraph end and makes LaTeX (re-)compute the line spacing with \tiny
}

\blindtext

\subsubsection{A very very very very very very very very very very long title \texorpdfstring{\hyperref[tab:literature]{\footnotesize \hfill\faIcon[regular]{arrow-alt-circle-up}}}{}}
% \texorpdfstring is used to generate proper PDF indices
\label{lit:labelOfYourSubsubsection5}		% The \label should always be after \caption
{
	\tiny
	Authors: Dana Vantrease, Robert Schreiber, Matteo Monchiero, Moray McLaren, Norman P. Jouppi, Marco Fiorentino, Al Davis, Nathan Binkert, Raymond G. Beausoleil, Jung Ho Ahn \cite{Something}
	\par  % \par is the paragraph end and makes LaTeX (re-)compute the line spacing with \tiny
}

\cite{Pan, Something, HK, Boney96}
\blindtext

% Resume to show up to \subsubsection in the ToC
\addtocontents{toc}{\protect\setcounter{tocdepth}{3}}
% -----------------------------------------------------------------------------
%              End of section "Optical Inter/Intra-Chip Networks"
% -----------------------------------------------------------------------------

\newpage
\begin{appendices}
	\section{An Appendix}

	% Include another tex file
	% A header is needed on top of the subfile if you want to compile it separately
	\subfile{latex/classification.tex}

	% The hyperref anchor for key is named cite.key. It is an anchor, not a label, so you have to use \hyperlink instead of \hyperref, but that doesn't use the correct link border color etc. 
	%\hyperlink{cite.chen_sharing_2014}{Sharing and placement}

	\section{Another Appendix}
	
	\subsection{A Subection in Appendix}
	
	\section{\href{\myBlogUrl}{\noindent My Blog \texorpdfstring{\small \faIcon{external-link-alt}}{}}}
	
\end{appendices}

% Uncomment this to use your own .Bib file
%\bibliography{bibliography/yourBibFile}

% A dummy bibliography list. 100 is a random guess of the total number of references
\begin{thebibliography}{100}
	
	\bibitem{Boney96} Boney, L., Tewfik, A.H., and Hamdy, K.N., ``Digital
	Watermarks for Audio Signals," \emph{Proceedings of the Third IEEE
		International Conference on Multimedia}, pp. 473-480, June 1996.
	
	\bibitem{MG} Goossens, M., Mittelbach, F., Samarin, \emph{A LaTeX
		Companion}, Addison-Wesley, Reading, MA, 1994.
	
	\bibitem{HK} Kopka, H., Daly P.W., \emph{A Guide to LaTeX},
	Addison-Wesley, Reading, MA, 1999.
	
	\bibitem{Pan} Pan, D., ``A Tutorial on MPEG/Audio Compression," \emph{IEEE
		Multimedia}, Vol.2, pp.60-74, Summer 1998.
	
	\bibitem{Something} Shixi, ``A Better Way to Include Bibliography is to Use Zotera," \emph{IEEE
		}, Vol.2, pp.60-74, Summer 2021.
	
\end{thebibliography}

\end{document}   