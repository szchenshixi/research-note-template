% !TeX root = chapter_template.tex
% !TeX spellcheck = en_US
\documentclass[../template.tex]{subfiles}

\begin{document}
\section{Your Section One}

\blindtext

\begin{formal}
	\textbf{Q}: How can I place a continuous discussion thread that tracks my thinking process?
	
	\noindent\textbf{A}: Write your thinking in a Q\&A format inside a \textbackslash begin\{formal\} \textbackslash\{formal\} block. 
\end{formal}

\subsection{Your Subsection}

Ensuring a low enough bit error rate (BER) at the receiver requires that the received signal must exceed the minimum detectable power of the receiver. This requirement can be expressed mathematically as
\begin{equation}
	P_{mod}-L_{IO}-L_{IC} \geq P_{min,det}, \tagaddtext{[dBm]}
	\label{eqn:opticalPowerBudget}
\end{equation}
where $P_{mod}$ represents the signal power immediately after modulation, $L_{IO}$ represents the coupling loss at chip boundaries, and $L_{IC}$ represents the on-chip loss due to devices such as waveguides, fibers, and micro-ring resonators (MRRs). The parameter $P_{min,det}$ denotes the minimum detectable power, which is by definition the receiver's sensitivity and is expressed in decibels relative to one milliwatt (dBm).

\begin{figure}[!hbt]
	\centering
	\begin{subfigure}[b]{0.38\textwidth}
		\includegraphics[width=\textwidth]{example-image-a}
		\caption{}
		\label{fig:exampleImageA}		% The \label should always be after \caption
		\includegraphics[width=\textwidth]{example-image-b}
		\caption{}
		\label{fig:exampleImageB}		% The \label should always be after \caption
	\end{subfigure}
	\begin{subfigure}[b]{0.6\textwidth}
		\centering
		\includegraphics[width=\textwidth,angle=90]{example-image-c}
		\caption{}
		\label{fig:exampleImageC}		% The \label should always be after \caption
	\end{subfigure}
	\caption{Explanation of the figure presentation. (a) Evidence A. (b) Evidence B. (c) Evidence C.}
	\label{fig:exampleImages}		% The \label should always be after \caption
\end{figure}

\blindtext


\begin{figure}
	\centering
	\begin{elasticrow}[width=.98\linewidth, sep=1em]
		\elasticfigure[Aspect ratio 16x9]{example-image-16x9}
		\elasticfigure[Aspect ratio 4x3]{example-image-4x3}
	\end{elasticrow}
	\caption{An example of aligning several images in one row so that they have the same height and, at the same time, fill a specified width, with an optionally specified white space between.}
\end{figure}

\begin{noteblock}
	Some information goes here
	
	And some other information goes here
\end{noteblock}

\blindtext

\begin{table}[!htb]
	\centering
	%	\renewcommand{\arraystretch}{1.5}
	\caption{Electrical Power Model}
	\label{tab:electricalPwrModel}		% The \label should always be after \caption
	\begin{adjustbox}{max width=\textwidth}  % Make the table fit text width
		\begin{threeparttable}
			\begin{tabular}{l r r r}
				\toprule
				\textbf{Parameter} 					  & \textbf{Dynamic Power}    & \textbf{Static Power} 			& \textbf{Note}					 \\ \midrule
				core@4GHz + L1 I/D Cache 			  & 1.50185 W  			      & 0.131346 W 		   				& \multirow[c]{7}{*}{McPAT 7nm}  \\
				% [vpos]{nrows}{width}
				\multirow[t]{2}{*}{512KB L2 Slice}    & Read: 0.0226595 nJ/op     & \multirow[t]{2}{*}{0.0238705 W} &            					 \\
				& Write: 0.0226595 nJ/op    &                                 &                                \\
				\multirow[t]{2}{*}{4MB L3 Slice}      & Read: 0.080728 nJ/op      & \multirow[t]{2}{*}{0.189787 W}  &                                \\
				& Write: 0.080728 nJ/op     &            						&								 \\
				router                                & 0.09 pJ/bit               & 0.11 W                          &                                \\
				buffer                                & 0.064 pJ/bit              & 0.0002 W                        &                                \\ \midrule
				intra-die elink (16-Byte wide, uni-)  & 0.0768 pJ/bit             & 0.017 W                         &                                \\
				inter-die elink                       & 1.02 pJ/bit               & \textcolor{red}{0 W, no report} & \cite{Something}               \\
				inter-chip elink \tnote{a}            & 4.71 pJ/bit               & \textcolor{red}{0 W, no report} & OEIL                           \\ \midrule
				memory controller                     & 0.0391783 W               & 0.00194085 W                    & McPAT                          \\ \midrule
				memory                                & 0.24 W                    & 0.075 W                         & MICRON DDR4 8GB                \\
				&                           &                                 & @(9\% read, 3\%write)          \\ \bottomrule
			\end{tabular}
			\begin{tablenotes}
				\newcommand{\CORNERSTONE}{\href{run:./appendix/CORNERSTONE-Standard-Components-Library-SOI.pdf}{CORNERSTONE}}
				\scriptsize
				\item [a] Highly dependent on length, working frequency, etc. Value derived from OEIL's default device parameters, check the manual for details.
			\end{tablenotes}
		\end{threeparttable}
	\end{adjustbox}
\end{table}

\subsubsection{Your Subsubsection}

\paragraph{Object1} is the focus of this paragraph. \blindtext

\paragraph{Object2} is the focus of this paragraph. \blindtext

\subsection{Literature}
% From this point on, hide sub-sub-sections in the ToC
\addtocontents{toc}{\protect\setcounter{tocdepth}{2}}

\cite{Boney96, MG, HK, Pan, Something} has reported promising results on phenomena A, B, C while \cite{Something} provides more evidence to support the hypothesis. Particularly, \citep[Sec.~III]{Something} explains their relationship in great details and proposes a new model called WWW.

\begin{table}[!htb]
	\centering
	\label{tab:literature}		% The \label should always be after \caption
	% \resizebox{\textwidth}{!}{  % Make the table fit text width
		\begin{adjustbox}{max width=\textwidth}  % Make the table fit text width
			\begin{threeparttable}
				\newcommand{\titlelength}{230pt}  % Defines the display length of a title
				\begin{tabular}{ l l l l l }
					\toprule
					\textbf{Title}                                                                                          & \textbf{Type} & \textbf{Date} & \textbf{Group} & \textbf{Page}                                  \\ \midrule
					\truncate{\titlelength}{\good A very very very very very very very very very very long title}\cite{Boney96}   & Note1         & 2008          & Boney96        & \ref{lit:labelOfYourSubsubsection1} \\
					\truncate{\titlelength}{\good A very very very very very very very very very very long title}\cite{MG}        & Note2         & 2009          & MG             & \ref{lit:labelOfYourSubsubsection2} \\
					\truncate{\titlelength}{\soso A very very very very very very very very very very long title}\cite{HK}        & Note3         & 2010          & HK             & \ref{lit:labelOfYourSubsubsection3} \\
					\truncate{\titlelength}{\good A very very very very very very very very very very long title}\cite{Pan}       & Note4         & 2011          & Pan            & \ref{lit:labelOfYourSubsubsection4} \\
					\truncate{\titlelength}{\bad A very very very very very very very very very very long title}\cite{Something}  & Note5         & 2012          & Shixi          & \ref{lit:labelOfYourSubsubsection5} \\ \bottomrule
				\end{tabular}
			\end{threeparttable}
		\end{adjustbox}
	\end{table}
	
\subsubsection{A very very very very very very very very very very long title \texorpdfstring{\hyperref[tab:literature]{\footnotesize \hfill\faIcon[regular]{arrow-alt-circle-up}}}{}}
% \texorpdfstring is used to generate proper PDF indices
\label{lit:labelOfYourSubsubsection1}		% The \label should always be after \caption
{
	\tiny
	Authors: Dana Vantrease, Robert Schreiber, Matteo Monchiero, Moray McLaren, Norman P. Jouppi, Marco Fiorentino, Al Davis, Nathan Binkert, Raymond G. Beausoleil, Jung Ho Ahn \cite{Boney96}
	\par  % \par is the paragraph end and makes LaTeX (re-)compute the line spacing with \tiny
}

\blindtext

\subsubsection{A very very very very very very very very very very long title \texorpdfstring{\hyperref[tab:literature]{\footnotesize \hfill\faIcon[regular]{arrow-alt-circle-up}}}{}}
% \texorpdfstring is used to generate proper PDF indices
\label{lit:labelOfYourSubsubsection2}		% The \label should always be after \caption
{
	\tiny
	Authors: Dana Vantrease, Robert Schreiber, Matteo Monchiero, Moray McLaren, Norman P. Jouppi, Marco Fiorentino, Al Davis, Nathan Binkert, Raymond G. Beausoleil, Jung Ho Ahn \cite{MG}
	\par  % \par is the paragraph end and makes LaTeX (re-)compute the line spacing with \tiny
}

\blindtext

\subfile{pseudocode.tex}

\subsubsection{A very very very very very very very very very very long title \texorpdfstring{\hyperref[tab:literature]{\footnotesize \hfill\faIcon[regular]{arrow-alt-circle-up}}}{}}
% \texorpdfstring is used to generate proper PDF indices
\label{lit:labelOfYourSubsubsection3}		% The \label should always be after \caption
{
	\tiny
	Authors: Dana Vantrease, Robert Schreiber, Matteo Monchiero, Moray McLaren, Norman P. Jouppi, Marco Fiorentino, Al Davis, Nathan Binkert, Raymond G. Beausoleil, Jung Ho Ahn \cite{HK}
	\par  % \par is the paragraph end and makes LaTeX (re-)compute the line spacing with \tiny
}

\blindtext

\subsubsection{A very very very very very very very very very very long title \texorpdfstring{\hyperref[tab:literature]{\footnotesize \hfill\faIcon[regular]{arrow-alt-circle-up}}}{}}
% \texorpdfstring is used to generate proper PDF indices
\label{lit:labelOfYourSubsubsection4}		% The \label should always be after \caption
{
	\tiny
	Authors: Dana Vantrease, Robert Schreiber, Matteo Monchiero, Moray McLaren, Norman P. Jouppi, Marco Fiorentino, Al Davis, Nathan Binkert, Raymond G. Beausoleil, Jung Ho Ahn \cite{Pan}
	\par  % \par is the paragraph end and makes LaTeX (re-)compute the line spacing with \tiny
}

\blindtext

\subsubsection{A very very very very very very very very very very long title \texorpdfstring{\hyperref[tab:literature]{\footnotesize \hfill\faIcon[regular]{arrow-alt-circle-up}}}{}}
% \texorpdfstring is used to generate proper PDF indices
\label{lit:labelOfYourSubsubsection5}		% The \label should always be after \caption
{
	\tiny
	Authors: Dana Vantrease, Robert Schreiber, Matteo Monchiero, Moray McLaren, Norman P. Jouppi, Marco Fiorentino, Al Davis, Nathan Binkert, Raymond G. Beausoleil, Jung Ho Ahn \cite{Something}
	\par  % \par is the paragraph end and makes LaTeX (re-)compute the line spacing with \tiny
}

\cite{Pan, Something, HK, Boney96}
\blindtext

% Resume to show up to \subsubsection in the ToC
\addtocontents{toc}{\protect\setcounter{tocdepth}{3}}
% -----------------------------------------------------------------------------
%              End of chapter "Optical Inter/Intra-Chip Networks"
% -----------------------------------------------------------------------------
\end{document}