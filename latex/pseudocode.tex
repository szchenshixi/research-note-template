% !TeX root = pseudocode.tex
% !TeX spellcheck = en_US
\providecommand{\projectroot}{..}  % bibtex will need this to find bibliography files
\documentclass[\projectroot/template.tex]{subfiles}

%\usepackage{amsmath}
%\usepackage{algorithm}
%\usepackage[noend]{algpseudocode}

\begin{document}
\makeatletter
\def\BState{\State\hskip-\ALG@thistlm}
\makeatother
\begin{algorithm}
	\caption{My algorithm}\label{alg:euclid1}
	\begin{algorithmic}[1]
		\Procedure{MyProcedure}{}
		\State $\textit{stringlen} \gets \text{length of }\textit{string}$
		\State $i \gets \textit{patlen}$
		\BState \emph{top}:
		\If {$i > \textit{stringlen}$} \Return false
		\EndIf
		\State $j \gets \textit{patlen}$
		\BState \emph{loop}:
		\If {$\textit{string}(i) = \textit{path}(j)$}
		\State $j \gets j-1$.
		\State $i \gets i-1$.
		\State \textbf{goto} \emph{loop}.
		\State \textbf{close};
		\EndIf
		\State $i \gets i+\max(\textit{delta}_1(\textit{string}(i)),\textit{delta}_2(j))$.
		\State \textbf{goto} \emph{top}.
		\EndProcedure
	\end{algorithmic}
\end{algorithm}
	
\section{Example Algorithm}

Algorithms can be included using the commands as shown in algorithm \ref{alg:euclid1}.

\begin{algorithm}
	\caption{Euclid’s algorithm}\label{alg:euclid2}
	\textbf{INPUT:} $x$ - input data\\
	\textbf{OUTPUT:} $abc$ is $x$
	\begin{algorithmic}[1]
		\Procedure{Euclid}{$a,b$}\Comment{The g.c.d. of a and b}
		\State $r\gets a\bmod b$
		\While{$r\not=0$}\Comment{We have the answer if r is 0}
		\State $a\gets b$
		\State $b\gets r$
		\State $r\gets a\bmod b$
		\EndWhile\label{euclidendwhile}
		\State \textbf{return} $b$\Comment{The gcd is b}
		\EndProcedure
	\end{algorithmic}
\end{algorithm}
\end{document} 